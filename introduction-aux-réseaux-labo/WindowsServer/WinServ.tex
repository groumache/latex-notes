% !TEX root = ../Network.tex



\documentclass[class=article, crop=false]{standalone}
\usepackage[subpreambles=true]{standalone}
\usepackage{import}

%% TO BE COMMENTED
\usepackage{xcolor}
\usepackage{amsmath}
% \usepackage{mdframed}
% \newmdenv[topline=false, bottomline=false, rightline=false, skipabove=\topsep, skipbelow=\topsep]{example}
% \usepackage{tikz}
% \usetikzlibrary{shapes.geometric, calc, arrows}


\begin{document}















\section{Windows Server}





\textbf{Attention !}
\begin{itemize}
    \item Il faut configurer les paramètres réseaux du Windows Server en mode statique.
    \item Si on configure le serveur DHCP, il faut que ce soit dans un réseau interne.
    \item Le DNS du Windows Server est sa propre adresse IP.
\end{itemize}















\subsection{Initialisation} \label{subsec:WinServInstallation}





\begin{itemize}





%% Ctrl+Alt+Delete
\item Pour déverrouiller Windows Server, on utilise \texttt{Ctrl+Alt+Delete}. Pour le faire, dans le menu en haut à gauche de la fenêtre, clicker sur \textit{Entrée}, puis sur \textit{Clavier}, puis \textit{Envoyer Ctrl-Alt-Del}.





%% Installation des rôles
\item Pour installer les rôles (webserveur, dns, dhcp), on va:
\begin{enumerate}
    \item dans la fenêtre \textit{Server Manager}, en haut à droite, clicker sur \textit{Manage}, puis sur \textit{Add Roles and Features}
    \item dans \textit{Installation Type} et \textit{Server Selection}, laisser les paramètres par défaut
    \item dans \textit{Server Roles}, ajouter les rôles:
    \begin{itemize}
        \item DHCP Server
        \item DNS Server
        \item Web Server (IIS)
    \end{itemize}
    \item pour le reste, garder les paramètres par défaut (clicker sur \textit{Next}), puis clicker sur \textit{Install}.
    \item une fois l'installation terminée, clicker sur le drapeau en haut à droite, puis sur \textit{Complete DHCP configuration}.
\end{enumerate}





%% Configuration Client
\item Quelle configuration pour le client ?
\begin{itemize}
    \item Si on a un serveur DHCP, on le laisse en DHCP.
    \item Sinon, on peut choisir entre le mettre en DHCP ou en statique. \textbf{Attention !} Il faut configurer le serveur manuellement quand même.
\end{itemize}





\end{itemize}















\subsection{Configuration DHCP} \label{subsec:WinServDHCP}





\begin{itemize}





%% Configuration DHCP
\item Pour configurer le serveur DHCP:
\begin{enumerate}
    \item dans la fenêtre \textit{Server Manager}, en haut à droite, clicker sur \textit{Tools}, puis sur \textit{DHCP}
    \item dans la fenêtre \textit{DHCP}, dans le menu de gauche, click-droit sur \textit{IPv4}
    \item clicker sur \textit{New Scope}
    \item nom du \textit{Scope} = pas important
    \item choisir le range d'adresses ip à distribuer, ainsi que les plages d'exclusions
    \item dans \textit{Configure DHCP Options}, sélectionner \textit{Yes, I want to configure these options now}
    \item donner l'adresse du serveur DNS (= adresse Windows Server).
\end{enumerate}





%% Réservation d'adresse IP
\item Pour réserver une adresse IP pour une MAC address particulière, il faut:
\begin{enumerate}
    \item dans la fenêtre \textit{DHCP}, dans le menu de gauche, click-droit sur \textit{IPv4/Scope/Reservations}
    \item clicker sur \textit{New Reservation}.
\end{enumerate}





\end{itemize}















\subsection{Créer un 1er site web} \label{subsec:WinServSite}





\begin{itemize}





%% DNS
\item Configuration DNS:
\begin{enumerate}
    \item \textbf{Forward Lookup Zone}
    \begin{enumerate}
        \item dans la fenêtre \textit{Server Manager}, en haut à droite, clicker sur \textit{Tools}, puis sur \textit{DNS}
        \item dans le menu de gauche de la fenêtre \textit{DNS Manager}, en dessous du nom du serveur (Win-DI...), click-gauche puis click-droit sur \textit{Forward Lookup Zone}
        \item clicker sur \textit{New Zone}
        \item sélectionner \textit{Primary Zone}, puis mettre le nom de domaine (ex: \texttt{greg.labo})
        \item puis continuer avec les paramètres par défaut.
    \end{enumerate}
    \item \textbf{Reverse Lookup Zone}
    \begin{enumerate}
        \item Ensuite, dans \textit{DNS Manager}, click-gauche puis click-droit sur \textit{Reverse Lookup Zone}
        \item clicker sur \textit{New Zone}
        \item sélectionner \textit{Primary Zone}, puis \textit{IPv4 Reverse Lookup Zone}
        \item dans \textit{Network ID}, mettre le réseau dans lequel le serveur opère
        \item puis continuer avec les paramètres par défaut.
    \end{enumerate}
    \item \textbf{Création d'un \textcolor{red}{enregistrement A}} (\texttt{greg.labo} $ \rightarrow $ \texttt{172.10.0.20})
    \begin{enumerate}
        \item dans la fenêtre \textit{DNS Manager}, dans le menu de gauche, click-droit sur le nom de votre site (ex: \texttt{greg.labo}) et clicker sur \textit{New Host (A or AAAA)}.
        \item dans \textit{Name}, on peut mettre ce qu'on veut, voir rien du tout
        \item dans \textit{IP address}, mettre l'adresse du serveur
        \item cocher la case \textit{Create associated pointer (PTR) record}.
    \end{enumerate}
    \item \textbf{Création d'un \textcolor{red}{enregistrement CNAME}} (\texttt{www.greg.labo} $ \rightarrow $ \texttt{greg.labo})
    \begin{enumerate}
        \item dans la fenêtre \textit{DNS Manager}, dans le menu de gauche, click-droit sur le nom de domaine (ex: \texttt{greg.labo})
        \item clicker sur \textit{New Alias (CNAME)}
        \item utiliser \textit{www} comme \textit{Alias Name}
        \item au lieu d'écrire dans \textit{FQDN for target host}, clicker sur \textit{Browse}
        \item sélectionner l'enregistrement A que vous avez créé précédemment.
    \end{enumerate}
    \item \textbf{Remarque}: \\
    on n'est pas obligé de créer un enregistrement CNAME, mais (apparemment) c'est une bonne pratique de créer un enregistrement: \texttt{www.greg.labo}, qui pointe vers: \texttt{greg.labo}, ou l'inverse en fonction de l'enregistrement A.
\end{enumerate}





%% IIS
\item Configuration WebServer (\textbf{IIS}):
\begin{enumerate}
    \item dans la fenêtre \textit{Server Manager}, en haut à droite, clicker sur \textit{Tools}, puis sur \textit{Internet Information Services (IIS) Manager}
    \item dans cette nouvelle fenêtre, dans le menu de gauche, click-droit sur \textit{Sites}, puis clicker sur \textit{Add Website}
    \item remplir le formulaire:
    \begin{itemize}
        \item \textit{Site name} -- pas important
        \item \textit{Physical path} -- créer un nouveau dossier dans: \texttt{C:$\backslash$inetpub$\backslash$wwwroot$\backslash$}
        \item \textit{IP address} -- adresse du serveur
        \item \textit{Host name} -- \texttt{www.greg.labo}
    \end{itemize}
    \item à l'endroit indiqué dans \textit{Physical path}, créer un document \textit{index.html}, ce sera la première page du site web.
\end{enumerate}





%% Sites Bindings
\item On peut lier un \textbf{deuxième} nom de domaine au site en allant dans:
\begin{itemize}
    \item \textit{IIS Manager}, dans le menu de gauche, click-droit sur le nom du site web
    \item clicker sur \textit{Edit Bindings}.
\end{itemize}
C'est utile pour que: \texttt{www.greg.labo}, et: \texttt{greg.labo}, affichent le même site.





\end{itemize}















\subsection{Créer un 2ème site}





\begin{itemize}





\item Ajouter un enregistrement CNAME $ \implies $ affiche le site par défaut.





\item \textbf{Lier} un nouvel enregistrement CNAME $ \implies $ affiche le premier site.





\item Créer un nouveau site $ \implies $ affiche le nouveau site.
\begin{example}
    Pareil que créer le premier site, excepté pour la reverse zone, c-à-d:
    \begin{enumerate}
        \item créer une \textit{forward lookup zone}
        \item \textbf{ne pas} créer une \textit{reverse lookup zone}
        \item créer un \textit{enregistrement A}
        \item potentiellement, créer un \textit{enregistrement CNAME}.
    \end{enumerate}
\end{example}





\end{itemize}















\subsection{Connection HTTPS}





\begin{itemize}





%% DNS
\item Ajouter une entrée DNS: créer un enregistrement CNAME pour: \texttt{secure.greg.labo}.





%% Certificat auto-signé
\item Créer un certificat auto-signé:
\begin{enumerate}
    \item ouvrir la fenêtre \textit{IIS Manager}
    \item dans le menu de gauche, clicker sur le nom du server (= Win-DI...)
    \item dans le menu central (dans la section IIS), click-droit sur \textit{Server Certificates}
    \item clicker sur \textit{Open Feature}
    \item dans le menu de droite, clicker sur \textit{Create Self-Signed Certificate}
    \item choisir un nom et sélectionner \textit{Web Hosting}
\end{enumerate}





%% HTTPS
\item Ajouter la liaison sécurisée:
\begin{enumerate}
    \item dans la fenêtre \textit{IIS Manager}
    \item dans le menu de gauche, click-droit sur le site pour lequel vous voulez la liaison sécurisée
    \item clicker sur \textit{Edit Bindings}, puis sur \textit{Add}
    \item dans \textit{Type}, mettre \texttt{https}, puis mettre l'IP du serveur
    \item dans \textit{Host Name}, mettre \texttt{secure.greg.labo}
    \item sélectionner le certificat créé précédemment.
\end{enumerate}





%% Remarque
\item \textbf{Attention !} Il faut absolument mettre \textbf{https} dans le navigateur (\texttt{https://secure.greg.labo}).





\end{itemize}















\subsection{Serveur FTP}





\begin{itemize}





%% Rôles
\item Pour mettre en place un serveur FTP, il faut installer les rôles: \\
\begin{tabular}{p{6cm}p{6cm}}
\begin{itemize}
    \item Web Server (IIS)
    \item FTP Server
\end{itemize}
&
\begin{itemize}
    \item FTP Service
    \item FTP Extensability
\end{itemize}
\end{tabular} \\
Remarque: ils sont tous les uns en dessous des autres dans \textit{Server Roles}.





%% Site FTP
\item Créer un site FTP:
\begin{enumerate}
    \item ouvrir la fenêtre \textit{IIS Manager}
    \item dans le menu de gauche, click-droit sur \textit{site FTP}
    \item clicker sur \textit{Add FTP Site}, puis mettre le nom du serveur
    \item créer un répertoire pour le site web (ex: \texttt{C:$\backslash$inetpub$\backslash$wwwroot$\backslash$FTP-Folder})
    \item indiquer l'adresse IP du serveur et désactiver le SSL
    \item mettre: \textit{Authentification} = \textit{Basic}, \textit{Authorization} = \textit{All users}
    \item cocher les cases \textit{Read} et \textit{Write}.
\end{enumerate}





%% Accès au site FTP
\item Pour accéder au site FTP depuis le client, il faut:
\begin{enumerate}
    \item désactiver le firewall du serveur
    \item dans un navigateur ou un explorateur de fichiers, mettre: \texttt{ftp://<ip\_serveur>/}, dans la barre de recherche
    \item se connecter avec:
    \begin{itemize}
        \item login: \texttt{Administrator}
        \item mdp: \texttt{Tigrou007}
    \end{itemize}
\end{enumerate}





\end{itemize}










































































\subsection{Forwarder \& Conditional Forwarder}





\begin{itemize}





%% Topologie
\item Topologie:
\begin{center}
\begin{tikzpicture}
    %% réseau interne henallux
    \draw[-, very thick, red] (0,0) -- (11.5,0) node[anchor=west]{Réseau interne hénallux};

    %% internet
    \node (internet) [] at (9,1) {\textbf{Internet}};

    %% Windows 10 -- Greg
    \node (win10greg) [rectangle, anchor=south, minimum height=2cm, text centered, text width=2cm, fill=sprinen] at (6,1) {Windows 10 \\ Greg};

    %% Windows 10 -- autre
    \node (win10autre) [rectangle, anchor=north, minimum height=2cm, text centered, text width=2cm, fill=red!40] at (10,-1) {Windows 10 \\ autre};

    %% WinServ -- Greg
    \node (winservgreg) [anchor=south, fill=sprinen, text centered, rounded corners, diamond, aspect=2] at (2,1) {WinServ 2019 Greg};

    %% WinServ -- autre
    \node (winservautre) [anchor=north, fill=red!40, text centered, rounded corners, diamond, aspect=2] at (5,-1) {WinServ 2019 autre};


    %% links
    \draw[-, red] (internet) -- ($(internet) + (0,-1)$);
    \draw[-, red] (win10greg.south) -- ($(win10greg.south) + (0,-1)$);
    \draw[-, red] (winservgreg.south) -- ($(winservgreg.south) + (0,-1)$);
    \draw[-, red] (win10autre.north) -- ($(win10autre.north) + (0,1)$);
    \draw[-, red] (winservautre.north) -- ($(winservautre.north) + (0,1)$);


    %% www.greg.local
    \draw [->, blue] (win10autre.south) to [out=-150, in=-30] node[anchor=south, yshift=0.2cm, xshift=-0.2cm]{www.greg.labo} (winservautre.south);
    \draw [->, blue] ($(winservautre.west) + (0,0.1)$) to [out=160, in=-115] node[anchor=south, rotate=-55, xshift=0.2cm, yshift=0.2cm]{www.greg.labo} ($(winservgreg.south west) + (0.1,-0.1)$);

    %% dns response
    \draw [->, orange] ($(winservgreg.south west) + (-0.1,0)$) to [out=-115, in=160] node[anchor=north, rotate=-60, xshift=-0.1cm, yshift=-0.1cm]{adresse ip} ($(winservautre.west) + (0,-0.1)$);
    \draw [->, orange] ($(winservautre.south) + (0,-0.2)$) to [out=-30, in=-150] node[anchor=north, yshift=-0.1cm, xshift=-0.1cm]{adresse ip} ($(win10autre.south) + (0,-0.2)$);
\end{tikzpicture}
\end{center}





%% Serveur DHCP
\item \textbf{Attention !} Vu qu'on est connecté sur le réseau du laboratoire, il faut désactiver la fonction \textit{serveur dhcp} des serveurs windows.





%% Connexion Internet
\item Ajouter un \textit{Forwarder} pour la connexion à internet:
\begin{enumerate}
    \item dans la fenêtre \textit{Server Manager}, en haut à droite, clicker sur \textit{Tools}, puis sur \textit{DNS}
    \item dans la fenêtre, \textit{DNS Manager}, dans le menu de gauche, clicker sur \textit{Win-DI...} (= nom du serveur)
    \item click-droit sur \textit{Forwarder}, puis clicker sur \textit{Properties}
    \item clicker sur \textit{Edit} et ajouter les adresses des serveurs DNS de l'école: 10.101.210.\{8,9\}.
\end{enumerate}





%% Connexion: www.greg.labo
\item Connexion à un site web (\texttt{www.greg.labo}) hébergé sur le réseau interne avec un \textit{Conditional Forwarder}:
\begin{enumerate}
    \item dans la fenêtre \textit{Server Manager}, en haut à droite, clicker sur \textit{Tools}, puis sur \textit{DNS}
    \item dans la fenêtre, \textit{DNS Manager}, dans le menu de gauche, click-gauche puis click-droit sur \textit{Conditional Forwarders}
    \item clicker sur \textit{New Conditional Forwarder}
    \item dans \textit{DNS Domain}, mettre le nom de domaine auquel on veut accéder (ex: \texttt{greg.labo})
    \item dans \textit{IP addresses of the master servers}, ajouter l'adresse ip du serveur dns qui sait résoudre le nom de domaine (ici, l'ip de: WinServ 2019 Greg).
\end{enumerate}





\end{itemize}





\end{document}

