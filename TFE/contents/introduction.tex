% LTeX: language=fr

\chapter{Introduction}

Ce projet participe à l'amélioration de l'analyse des menaces et des données forensiques dans l'entreprise. Afin d'y parvenir, ce travail commence par une analyse des besoins et une révision des concepts théoriques associés au processus de réponse à incident pour avoir une vue d'ensemble des processus. S'ensuit la recherche et la comparaison d'un ensemble d'outils afin de les sélectionner sur base de critères objectifs.

Après avoir sélectionné un nombre de logiciels et avoir revu les concepts théoriques, vient la conception de l'architecture d'une solution d'analyse de données forensiques, qui est isolée autant que possible afin de prévenir la propagation des menaces. Son implémentation est également abordée, ainsi qu'une réflexion sur son utilisation, mais aussi sur les procédures d'acquisition et d'analyse forensique et de menaces.

Le sujet abordé étant vaste, il est impossible de le couvrir dans son entièreté. Ce travail offre, bien qu'il ait rempli les objectifs initiaux, de nombreuses possibilités d'amélioration et de sujets à approfondir. Ces pistes d'amélioration et sujets à approfondir seront donc aussi envisagés.

