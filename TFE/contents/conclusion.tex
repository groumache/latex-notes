% LTeX: language=fr

\chapter{Conclusion}

La solution qui a été créée pourra être utilisée pour analyser les tentatives de pénétration avec des mails de phishing ou des exécutables de manière automatique. Ceci fera gagner du temps aux analystes et leur permettra de rentrer plus en profondeur dans l'analyse ce qui permet de mieux prévenir les menaces. Au niveau de l’analyse forensique des systèmes contaminés, j’ai pu répondre aux attentes en termes de procédures et de la sélection d’outils, que j’ai d’ailleurs pu tester dans des situations réelles.

Bien que les objectifs du projet aient été atteints, il reste beaucoup d'améliorations possibles à apporter et de pistes à approfondir tant au niveau de l'analyse de malware que de l'analyse forensique. Parmi les améliorations, on compte des gains de temps qui peuvent être réalisés avec encore plus d'automatisation. Ce temps libéré peut alors être utilisé afin d'étudier plus en profondeur certains cas avec une analyse manuelle plus poussée.

