% LTeX: language=fr

\chapter*{Présentation de l'entreprise} \addcontentsline{toc}{chapter}{Présentation de l'entreprise}

\textit{Network Research Belgium}, souvent abrégé sous la forme du sigle \textit{NRB} est une entreprise belge du secteur des TIC, les technologies de l'information et de la communication, fondée en 1987 en Belgique, mais à vocation européenne. L'entreprise fournit des services dans les quatre domaines suivants:
\begin{itemize}
    \item la \textit{consultance} pour accompagner ses clients dans leurs démarches de transformation numérique et le conseil en cybersécurité;
    \item les \textit{services logiciels} comme le développement et la maintenance d'applications;
    \item les \textit{services infrastructures et clouds}, NRB permet à ses clients d'utiliser le cloud privé de NRB et les clouds publics d'IBM, Microsoft, Google et Amazon;
    \item les \textit{services de managed staffing} consistent à fournir du personnel à des entreprises qui en ont besoin pour conduire à bien leurs projets.
\end{itemize}

Certains de ces services sont fournis par les filiales du groupe NRB. En effet, NRB est un groupe qui s'agrandit notamment par des acquisitions d'entreprises dans le domaine technologique. En 2020, le groupe NRB comptait 3200 collaborateurs, avec un chiffre d'affaires de 413 millions d'euros.

Au sein de NRB, il y a l'équipe SecOps où j'ai réalisé mon stage. SecOps vient de \textit{Sécurité et Opérations}, c'est l'équipe principale de gestion de la sécurité chez NRB. Comme l'indique le mot \textit{sécurité} de SecOps, ils doivent gérer les incidents de sécurité qui se produisent dans l'entreprise et chez certains de leurs clients mais ils ne gèrent pas l'ensemble des \textit{opérations}. Par exemple, ce n'est pas leur rôle de conduire un changement réseau du début à la fin, c'est le rôle de l'équipe réseau. Cependant, ils ont la responsabilité de vérifier que ce changement ne comporte pas de risque de sécurité important.

